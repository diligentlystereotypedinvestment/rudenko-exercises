\documentclass{amsart}
\input{~/templates/preamble}
\usepackage[backend=biber]{biblatex}
\addbibresource{~/\$BIB}

\title{Lecture Exercises}
\author{Vincent Tran}

\begin{document}

\begin{abstract}
	A collection of exercises mentioned by Rudenko in lecture that test your understanding but aren't significant enough to be a true ``problem''.
	All should be done at least once.
	They tend to be more computational.
	If you can't do it, you can probably open up an old algebra textbook and find it.
	Dummit and Foote is enormous so it probably contains it.
\end{abstract}

\maketitle

\begin{exercise}
	Show that the identity and inverse of a group are unique.
	In addition, associativitiy holds for any number of elements, i.e. $g_{1}g_{2}g_{3}\cdots g_n $ is well-defined no matter the bracket placement.
\end{exercise}

\begin{exercise}
	Classify all the groups with three elements.
	\ifhint
		There is one.
	\fi
\end{exercise}

\begin{exercise}
	Show that if $f: G\to G $ is a function from a group $G $ to itself is such that $f(ab) = f(a) f(b) $, then $f(e) = e $ and $f(a^{-1}) = f(a)^{-1} $.
\end{exercise}

\begin{exercise}
	Show that $\S_3 $ is isomorphic to $\GL_2(\F_2) $.
\end{exercise}

\begin{exercise}
	Compute some operations in $S_n $.
\end{exercise}

\begin{exercise}
	Show that $A_5 $ is the smallest non-abelian group.

	Also show that $\Z / p\Z $ is a simple group.
\end{exercise}

\begin{exercise}
	Show that the multiplication operation for the quotient group ($(g_{1}N,g_{2}N) \mapsto g_{1}g_{2}N $) is well-defined.
\end{exercise}
\end{document}
